\documentclass {article}
\usepackage{fullpage}

\begin{document}

~\vfill
\begin{center}
\Large

A5 Project Proposal

Title:

Name: Srushti Koorapati

Student ID: 20389553

User ID: skoorapa
\end{center}
\vfill ~\vfill~
\newpage
\noindent{\Large \bf Final Project:}
\begin{description}
\item[Purpose]:\\
Make a webgl game which utilizes 10 different graphics concepts we learned in class. 
In the game designed, the user will be guiding a transportation vehicle through a field which spawns random trees. 
The user has to dodge the tress and get as far as possible. The user can move left right or jump. 
As the game progresses, the vehicle moves faster and more trees spawn, thus making it more difficult for the user to continue.

\item[Statement]:\\
For Ray Tracers: Paragraph describing interesting scene to be
rendered and what features are needed to achieve
this scene.

Paragraph: What it's about.

Paragraph: What to do.

Paragraph: Why it is interesting and challenging.

Paragraph: What I will learn

\item[Technical Outline]:\\
Basically, your objectives in your objective list should be fairly
short statements of the objective; you should provide additional
details about your objectives in this section to clarify what you
plan to do.

Further, survey the important data structures and algorithms that
will be necessary to achieve the goals, and (for ray tracing
projects) lists the new commands
that will need to be added to the input language.

To  get  bold face: {\bf bold face words}.  To get italics: {\it italic
face words}.  To  get typewriter font: {\tt typed words}.  To get
larger  words:  {\large large  words}.   To  get smaller words: 
{\small small words}.  

\item[Bibliography]:\\
Articles  and/or  books  with  important  information on the
topics of the project.

\end{description}
\newpage


\noindent{\Large\bf Objectives:}

{\hfill{\bf Full UserID:\rule{2in}{.1mm}}\hfill{\bf Student ID:\rule{2in}{.1mm}}\hfill}

\begin{enumerate}
\item[\_\_\_ 1:]  Objective one.

\item[\_\_\_ 2:]  Objective two.

\item[\_\_\_ 3:]  Objective three.

\item[\_\_\_ 4:]  Objective four.

\item[\_\_\_ 5:]  Objective five.

\item[\_\_\_ 6:]  Objective six.

\item[\_\_\_ 7:]  Objective seven.

\item[\_\_\_ 8:]  Objective eight.

\item[\_\_\_ 9:]  Objective nine.

\item[\_\_\_ 10:]  Objective ten.
\end{enumerate}

% Delete % at start of next line if this is a ray tracing project
% A4 extra objective:
\end{document}
