\documentclass {article}
\usepackage{fullpage}

\begin{document}

~\vfill
\begin{center}
\Large

A5 Project Proposal

Title:

Name: Srushti Koorapati

Student ID: 20389553

User ID: skoorapa
\end{center}
\vfill ~\vfill~
\newpage
\noindent{\Large \bf Final Project:}
\begin{description}
\item[Purpose]:\\
Make a webgl game which utilizes 10 different graphics concepts we learned in class. 
In the game designed, the user will be guiding a transportation vehicle through a field which spawns random trees. 
The user has to dodge the tress and get as far as possible. The user can move left right or jump. 
As the game progresses, the vehicle moves faster and more trees spawn, thus making it more difficult for the user to continue.

\item[Statement]:\\
The game is about a vehicle just attempting to dodge trees and move down the track into infinity. With each obstacle avoided it will become more challenging to continue surviving.   

The scene rendered will contain a sun (main light source), trees and the vehicle . The tress will likely be composed of a pyramid on a rectangular prism.
The vehicle will likely just be a block, this is to give the effect that the block is rubbing against the ground. This allows me to change the Terran to increase speed. 
The vehicle will move along a track and can be in a set amount of positions on the track. 

The interesting challenges come from how dynamic the scene will be. For starters everything will be moving. The main light source (the sun) will be in the rotating sky like a real sun. Eventually this sun will set. At this point the car must turn on its lights and now the user is limited to whatever light the spotlight will let them see. The changing of tracks will not simple either. Since the car takes a set amount of time to change the tracks, we have to account for dynamic collision as the car might not move to the right track in time to dodge. 

Since this is the first time I am developing a game, I will learn the architecture and challenges of developing a game. How shall I layout my variables? How can I get what I want and still give proper frame rate and many other challenges will be addressed. In terms of tools, I will be learning how to utilize WebGL and threejs framework for WebGL.

\item[Technical Outline]:\\
Basically, your objectives in your objective list should be fairly
short statements of the objective; you should provide additional
details about your objectives in this section to clarify what you
plan to do.

Further, survey the important data structures and algorithms that
will be necessary to achieve the goals, and (for ray tracing
projects) lists the new commands
that will need to be added to the input language.

To  get  bold face: {\bf bold face words}.  To get italics: {\it italic
face words}.  To  get typewriter font: {\tt typed words}.  To get
larger  words:  {\large large  words}.   To  get smaller words: 
{\small small words}.  

\item[Bibliography]:\\
Articles  and/or  books  with  important  information on the
topics of the project.

\end{description}
\newpage


\noindent{\Large\bf Objectives:}

{\hfill{\bf Full UserID:\rule{2in}{.1mm}}\hfill{\bf Student ID:\rule{2in}{.1mm}}\hfill}

\begin{enumerate}
\item[\_\_\_ 1:]  Objective one.

\item[\_\_\_ 2:]  Objective two.

\item[\_\_\_ 3:]  Objective three.

\item[\_\_\_ 4:]  Objective four.

\item[\_\_\_ 5:]  Objective five.

\item[\_\_\_ 6:]  Objective six.

\item[\_\_\_ 7:]  Objective seven.

\item[\_\_\_ 8:]  Objective eight.

\item[\_\_\_ 9:]  Objective nine.

\item[\_\_\_ 10:]  Objective ten.
\end{enumerate}

% Delete % at start of next line if this is a ray tracing project
% A4 extra objective:
\end{document}
